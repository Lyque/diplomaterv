%----------------------------------------------------------------------------
\chapter{Licencek eredeti sz�vegei}
%----------------------------------------------------------------------------

%----------------------------------------------------------------------------
\section{MIT License}
%----------------------------------------------------------------------------

\begin{Verbatim}[fontsize=\footnotesize]
The MIT License (MIT)
Copyright (c) <year> <copyright holders>

Permission is hereby granted, free of charge, to any person obtaining a 
copy of this software and associated documentation files (the "Software"), 
to deal in the Software without restriction, including without 
limitation the rights to use, copy, modify, merge, publish, distribute, 
sublicense, and/or sell copies of the Software, and to permit persons to 
whom the Software is furnished to do so, subject to the following 
conditions:

The above copyright notice and this permission notice shall be included 
in all copies or substantial portions of the Software.

THE SOFTWARE IS PROVIDED "AS IS", WITHOUT WARRANTY OF ANY KIND, EXPRESS 
OR IMPLIED, INCLUDING BUT NOT LIMITED TO THE WARRANTIES OF 
MERCHANTABILITY, FITNESS FOR A PARTICULAR PURPOSE AND NONINFRINGEMENT. 
IN NO EVENT SHALL THE AUTHORS OR COPYRIGHT HOLDERS BE LIABLE FOR ANY 
CLAIM, DAMAGES OR OTHER LIABILITY, WHETHER IN AN ACTION OF CONTRACT, 
TORT OR OTHERWISE, ARISING FROM, OUT OF OR IN CONNECTION WITH THE 
SOFTWARE OR THE USE OR OTHER DEALINGS IN THE SOFTWARE.
\end{Verbatim}

%----------------------------------------------------------------------------
\section{Apache License 2.0}
%----------------------------------------------------------------------------

\begin{Verbatim}[fontsize=\footnotesize]
Apache License
                           Version 2.0, January 2004
                        http://www.apache.org/licenses/

   TERMS AND CONDITIONS FOR USE, REPRODUCTION, AND DISTRIBUTION

   1. Definitions.

      "License" shall mean the terms and conditions for use, reproduction,
      and distribution as defined by Sections 1 through 9 of this document.

      "Licensor" shall mean the copyright owner or entity authorized by
      the copyright owner that is granting the License.

      "Legal Entity" shall mean the union of the acting entity and all
      other entities that control, are controlled by, or are under common
      control with that entity. For the purposes of this definition,
      "control" means (i) the power, direct or indirect, to cause the
      direction or management of such entity, whether by contract or
      otherwise, or (ii) ownership of fifty percent (50%) or more of the
      outstanding shares, or (iii) beneficial ownership of such entity.

      "You" (or "Your") shall mean an individual or Legal Entity
      exercising permissions granted by this License.

      "Source" form shall mean the preferred form for making modifications,
      including but not limited to software source code, documentation
      source, and configuration files.

      "Object" form shall mean any form resulting from mechanical
      transformation or translation of a Source form, including but
      not limited to compiled object code, generated documentation,
      and conversions to other media types.

      "Work" shall mean the work of authorship, whether in Source or
      Object form, made available under the License, as indicated by a
      copyright notice that is included in or attached to the work
      (an example is provided in the Appendix below).

      "Derivative Works" shall mean any work, whether in Source or Object
      form, that is based on (or derived from) the Work and for which the
      editorial revisions, annotations, elaborations, or other modifications
      represent, as a whole, an original work of authorship. For the purposes
      of this License, Derivative Works shall not include works that remain
      separable from, or merely link (or bind by name) to the interfaces of,
      the Work and Derivative Works thereof.

      "Contribution" shall mean any work of authorship, including
      the original version of the Work and any modifications or additions
      to that Work or Derivative Works thereof, that is intentionally
      submitted to Licensor for inclusion in the Work by the copyright owner
      or by an individual or Legal Entity authorized to submit on behalf of
      the copyright owner. For the purposes of this definition, "submitted"
      means any form of electronic, verbal, or written communication sent
      to the Licensor or its representatives, including but not limited to
      communication on electronic mailing lists, source code control systems,
      and issue tracking systems that are managed by, or on behalf of, the
      Licensor for the purpose of discussing and improving the Work, but
      excluding communication that is conspicuously marked or otherwise
      designated in writing by the copyright owner as "Not a Contribution."

      "Contributor" shall mean Licensor and any individual or Legal Entity
      on behalf of whom a Contribution has been received by Licensor and
      subsequently incorporated within the Work.

   2. Grant of Copyright License. Subject to the terms and conditions of
      this License, each Contributor hereby grants to You a perpetual,
      worldwide, non-exclusive, no-charge, royalty-free, irrevocable
      copyright license to reproduce, prepare Derivative Works of,
      publicly display, publicly perform, sublicense, and distribute the
      Work and such Derivative Works in Source or Object form.

   3. Grant of Patent License. Subject to the terms and conditions of
      this License, each Contributor hereby grants to You a perpetual,
      worldwide, non-exclusive, no-charge, royalty-free, irrevocable
      (except as stated in this section) patent license to make, have made,
      use, offer to sell, sell, import, and otherwise transfer the Work,
      where such license applies only to those patent claims licensable
      by such Contributor that are necessarily infringed by their
      Contribution(s) alone or by combination of their Contribution(s)
      with the Work to which such Contribution(s) was submitted. If You
      institute patent litigation against any entity (including a
      cross-claim or counterclaim in a lawsuit) alleging that the Work
      or a Contribution incorporated within the Work constitutes direct
      or contributory patent infringement, then any patent licenses
      granted to You under this License for that Work shall terminate
      as of the date such litigation is filed.

   4. Redistribution. You may reproduce and distribute copies of the
      Work or Derivative Works thereof in any medium, with or without
      modifications, and in Source or Object form, provided that You
      meet the following conditions:

      (a) You must give any other recipients of the Work or
          Derivative Works a copy of this License; and

      (b) You must cause any modified files to carry prominent notices
          stating that You changed the files; and

      (c) You must retain, in the Source form of any Derivative Works
          that You distribute, all copyright, patent, trademark, and
          attribution notices from the Source form of the Work,
          excluding those notices that do not pertain to any part of
          the Derivative Works; and

      (d) If the Work includes a "NOTICE" text file as part of its
          distribution, then any Derivative Works that You distribute must
          include a readable copy of the attribution notices contained
          within such NOTICE file, excluding those notices that do not
          pertain to any part of the Derivative Works, in at least one
          of the following places: within a NOTICE text file distributed
          as part of the Derivative Works; within the Source form or
          documentation, if provided along with the Derivative Works; or,
          within a display generated by the Derivative Works, if and
          wherever such third-party notices normally appear. The contents
          of the NOTICE file are for informational purposes only and
          do not modify the License. You may add Your own attribution
          notices within Derivative Works that You distribute, alongside
          or as an addendum to the NOTICE text from the Work, provided
          that such additional attribution notices cannot be construed
          as modifying the License.

      You may add Your own copyright statement to Your modifications and
      may provide additional or different license terms and conditions
      for use, reproduction, or distribution of Your modifications, or
      for any such Derivative Works as a whole, provided Your use,
      reproduction, and distribution of the Work otherwise complies with
      the conditions stated in this License.

   5. Submission of Contributions. Unless You explicitly state otherwise,
      any Contribution intentionally submitted for inclusion in the Work
      by You to the Licensor shall be under the terms and conditions of
      this License, without any additional terms or conditions.
      Notwithstanding the above, nothing herein shall supersede or modify
      the terms of any separate license agreement you may have executed
      with Licensor regarding such Contributions.

   6. Trademarks. This License does not grant permission to use the trade
      names, trademarks, service marks, or product names of the Licensor,
      except as required for reasonable and customary use in describing the
      origin of the Work and reproducing the content of the NOTICE file.

   7. Disclaimer of Warranty. Unless required by applicable law or
      agreed to in writing, Licensor provides the Work (and each
      Contributor provides its Contributions) on an "AS IS" BASIS,
      WITHOUT WARRANTIES OR CONDITIONS OF ANY KIND, either express or
      implied, including, without limitation, any warranties or conditions
      of TITLE, NON-INFRINGEMENT, MERCHANTABILITY, or FITNESS FOR A
      PARTICULAR PURPOSE. You are solely responsible for determining the
      appropriateness of using or redistributing the Work and assume any
      risks associated with Your exercise of permissions under this License.

   8. Limitation of Liability. In no event and under no legal theory,
      whether in tort (including negligence), contract, or otherwise,
      unless required by applicable law (such as deliberate and grossly
      negligent acts) or agreed to in writing, shall any Contributor be
      liable to You for damages, including any direct, indirect, special,
      incidental, or consequential damages of any character arising as a
      result of this License or out of the use or inability to use the
      Work (including but not limited to damages for loss of goodwill,
      work stoppage, computer failure or malfunction, or any and all
      other commercial damages or losses), even if such Contributor
      has been advised of the possibility of such damages.

   9. Accepting Warranty or Additional Liability. While redistributing
      the Work or Derivative Works thereof, You may choose to offer,
      and charge a fee for, acceptance of support, warranty, indemnity,
      or other liability obligations and/or rights consistent with this
      License. However, in accepting such obligations, You may act only
      on Your own behalf and on Your sole responsibility, not on behalf
      of any other Contributor, and only if You agree to indemnify,
      defend, and hold each Contributor harmless for any liability
      incurred by, or claims asserted against, such Contributor by reason
      of your accepting any such warranty or additional liability.

   END OF TERMS AND CONDITIONS
   
   
    APPENDIX: How to apply the Apache License to your work.

      To apply the Apache License to your work, attach the following
      boilerplate notice, with the fields enclosed by brackets "[]"
      replaced with your own identifying information. (Don't include
      the brackets!)  The text should be enclosed in the appropriate
      comment syntax for the file format. We also recommend that a
      file or class name and description of purpose be included on the
      same "printed page" as the copyright notice for easier
      identification within third-party archives.

   Copyright [yyyy] [name of copyright owner]

   Licensed under the Apache License, Version 2.0 (the "License");
   you may not use this file except in compliance with the License.
   You may obtain a copy of the License at

       http://www.apache.org/licenses/LICENSE-2.0

   Unless required by applicable law or agreed to in writing, software
   distributed under the License is distributed on an "AS IS" BASIS,
   WITHOUT WARRANTIES OR CONDITIONS OF ANY KIND, either express or implied.
   See the License for the specific language governing permissions and
   limitations under the License.
\end{Verbatim}

%----------------------------------------------------------------------------
\section{BSD}
%----------------------------------------------------------------------------

%----------------------------------------------------------------------------
\subsection{4-clause BSD (eredeti)}
%----------------------------------------------------------------------------

\begin{Verbatim}[fontsize=\footnotesize]
Copyright (c) <year> <copyright holder> . All rights reserved.
Redistribution and use in source and binary forms, with or without 
modification, are permitted provided that the following conditions are 
met:

1. Redistributions of source code must retain the above copyright notice, 
this list of conditions and the following disclaimer.

2. Redistributions in binary form must reproduce the above copyright 
notice, this list of conditions and the following disclaimer in the 
documentation and/or other materials provided with the distribution.

3. All advertising materials mentioning features or use of this software 
must display the following acknowledgement: 
This product includes software developed by the <organization>.

4. Neither the name of <copyright holder> nor the names of its 
contributors may be used to endorse or promote products derived from this 
software without specific prior written permission.

THIS SOFTWARE IS PROVIDED BY COPYRIGHT HOLDER "AS IS" AND ANY EXPRESS OR 
IMPLIED WARRANTIES, INCLUDING, BUT NOT LIMITED TO, THE IMPLIED WARRANTIES 
OF MERCHANTABILITY AND FITNESS FOR A PARTICULAR PURPOSE ARE DISCLAIMED. 
IN NO EVENT SHALL <COPYRIGHT HOLDER> BE LIABLE FOR ANY DIRECT, INDIRECT, 
INCIDENTAL, SPECIAL, EXEMPLARY, OR CONSEQUENTIAL DAMAGES (INCLUDING, BUT 
NOT LIMITED TO, PROCUREMENT OF SUBSTITUTE GOODS OR SERVICES; LOSS OF USE,
DATA, OR PROFITS; OR BUSINESS INTERRUPTION) HOWEVER CAUSED AND ON ANY 
THEORY OF LIABILITY, WHETHER IN CONTRACT, STRICT LIABILITY, OR TORT 
(INCLUDING NEGLIGENCE OR OTHERWISE) ARISING IN ANY WAY OUT OF THE USE OF 
THIS SOFTWARE, EVEN IF ADVISED OF THE POSSIBILITY OF SUCH DAMAGE.
\end{Verbatim}

%----------------------------------------------------------------------------
\subsection{3-clause BSD (m�dos�tott)}
%----------------------------------------------------------------------------

\begin{Verbatim}[fontsize=\footnotesize]
Copyright (c) <YEAR>, <OWNER>
All rights reserved.

Redistribution and use in source and binary forms, with or without 
modification, are permitted provided that the following conditions are 
met:

1. Redistributions of source code must retain the above copyright notice, 
this list of conditions and the following disclaimer.

2. Redistributions in binary form must reproduce the above copyright 
notice, this list of conditions and the following disclaimer in the 
documentation and/or other materials provided with the distribution.

3. Neither the name of the copyright holder nor the names of its 
contributors may be used to endorse or promote products derived from this 
software without specific prior written permission.

THIS SOFTWARE IS PROVIDED BY THE COPYRIGHT HOLDERS AND CONTRIBUTORS "AS 
IS" AND ANY EXPRESS OR IMPLIED WARRANTIES, INCLUDING, BUT NOT LIMITED TO, 
THE IMPLIED WARRANTIES OF MERCHANTABILITY AND FITNESS FOR A PARTICULAR 
PURPOSE ARE DISCLAIMED. IN NO EVENT SHALL THE COPYRIGHT HOLDER OR 
CONTRIBUTORS BE LIABLE FOR ANY DIRECT, INDIRECT, INCIDENTAL, SPECIAL, 
EXEMPLARY, OR CONSEQUENTIAL DAMAGES (INCLUDING, BUT NOT LIMITED TO, 
PROCUREMENT OF SUBSTITUTE GOODS OR SERVICES; LOSS OF USE, DATA, OR 
PROFITS; OR BUSINESS INTERRUPTION) HOWEVER CAUSED AND ON ANY THEORY OF 
LIABILITY, WHETHER IN CONTRACT, STRICT LIABILITY, OR TORT (INCLUDING 
NEGLIGENCE OR OTHERWISE) ARISING IN ANY WAY OUT OF THE USE OF THIS 
SOFTWARE, EVEN IF ADVISED OF THE POSSIBILITY OF SUCH DAMAGE.
\end{Verbatim}

%----------------------------------------------------------------------------
\subsection{2-clause BSD (egyszer�s�tett)}
%----------------------------------------------------------------------------

\begin{Verbatim}[fontsize=\footnotesize]
Copyright (c) <YEAR>, <OWNER>
All rights reserved.

Redistribution and use in source and binary forms, with or without 
modification, are permitted provided that the following conditions are 
met:

1. Redistributions of source code must retain the above copyright notice, 
this list of conditions and the following disclaimer.

2. Redistributions in binary form must reproduce the above copyright 
notice, this list of conditions and the following disclaimer in the 
documentation and/or other materials provided with the distribution.

THIS SOFTWARE IS PROVIDED BY THE COPYRIGHT HOLDERS AND CONTRIBUTORS "AS 
IS" AND ANY EXPRESS OR IMPLIED WARRANTIES, INCLUDING, BUT NOT LIMITED TO, 
THE IMPLIED WARRANTIES OF MERCHANTABILITY AND FITNESS FOR A PARTICULAR 
PURPOSE ARE DISCLAIMED. IN NO EVENT SHALL THE COPYRIGHT HOLDER OR 
CONTRIBUTORS BE LIABLE FOR ANY DIRECT, INDIRECT, INCIDENTAL, SPECIAL, 
EXEMPLARY, OR CONSEQUENTIAL DAMAGES (INCLUDING, BUT NOT LIMITED TO, 
PROCUREMENT OF SUBSTITUTE GOODS OR SERVICES; LOSS OF USE, DATA, OR 
PROFITS; OR BUSINESS INTERRUPTION) HOWEVER CAUSED AND ON ANY THEORY OF 
LIABILITY, WHETHER IN CONTRACT, STRICT LIABILITY, OR TORT (INCLUDING 
NEGLIGENCE OR OTHERWISE) ARISING IN ANY WAY OUT OF THE USE OF THIS 
SOFTWARE, EVEN IF ADVISED OF THE POSSIBILITY OF SUCH DAMAGE.
\end{Verbatim}

%----------------------------------------------------------------------------
\section{GNU GPL}
%----------------------------------------------------------------------------

%----------------------------------------------------------------------------
\subsection{GNU GPLv2}
%----------------------------------------------------------------------------

\begin{Verbatim}[fontsize=\footnotesize]
TERMS AND CONDITIONS FOR COPYING, DISTRIBUTION AND MODIFICATION

0. This License applies to any program or other work which contains a 
notice placed by the copyright holder saying it may be distributed under 
the terms of this General Public License. The "Program", below, refers to 
any such program or work, and a "work based on the Program" means either 
the Program or any derivative work under copyright law: that is to say, a 
work containing the Program or a portion of it, either verbatim or with 
modifications and/or translated into another language. (Hereinafter, 
translation is included without limitation in the term "modification".) 
Each licensee is addressed as "you".

Activities other than copying, distribution and modification are not 
covered by this License; they are outside its scope. The act of running 
the Program is not restricted, and the output from the Program is covered 
only if its contents constitute a work based on the Program (independent 
of having been made by running the Program). Whether that is true depends 
on what the Program does.

1. You may copy and distribute verbatim copies of the Program's source 
code as you receive it, in any medium, provided that you conspicuously 
and appropriately publish on each copy an appropriate copyright notice 
and disclaimer of warranty; keep intact all the notices that refer to 
this License and to the absence of any warranty; and give any other 
recipients of the Program a copy of this License along with the Program.

You may charge a fee for the physical act of transferring a copy, and you 
may at your option offer warranty protection in exchange for a fee.

2. You may modify your copy or copies of the Program or any portion of it, 
thus forming a work based on the Program, and copy and distribute such 
modifications or work under the terms of Section 1 above, provided that 
you also meet all of these conditions:

a) You must cause the modified files to carry prominent notices stating 
that you changed the files and the date of any change.
b) You must cause any work that you distribute or publish, that in whole 
or in part contains or is derived from the Program or any part thereof, 
to be licensed as a whole at no charge to all third parties under the 
terms of this License.
c) If the modified program normally reads commands interactively when run, 
you must cause it, when started running for such interactive use in the 
most ordinary way, to print or display an announcement including an 
appropriate copyright notice and a notice that there is no warranty (or 
else, saying that you provide a warranty) and that users may redistribute 
the program under these conditions, and telling the user how to view a 
copy of this License. (Exception: if the Program itself is interactive 
but does not normally print such an announcement, your work based on the 
Program is not required to print an announcement.)
These requirements apply to the modified work as a whole. If identifiable 
sections of that work are not derived from the Program, and can be 
reasonably considered independent and separate works in themselves, then 
this License, and its terms, do not apply to those sections when you 
distribute them as separate works. But when you distribute the same 
sections as part of a whole which is a work based on the Program, the 
distribution of the whole must be on the terms of this License, whose 
permissions for other licensees extend to the entire whole, and thus to 
each and every part regardless of who wrote it.

Thus, it is not the intent of this section to claim rights or contest 
your rights to work written entirely by you; rather, the intent is to 
exercise the right to control the distribution of derivative or 
collective works based on the Program.

In addition, mere aggregation of another work not based on the Program 
with the Program (or with a work based on the Program) on a volume of a 
storage or distribution medium does not bring the other work under the 
scope of this License.

3. You may copy and distribute the Program (or a work based on it, under 
Section 2) in object code or executable form under the terms of Sections 
1 and 2 above provided that you also do one of the following:

a) Accompany it with the complete corresponding machine-readable source 
code, which must be distributed under the terms of Sections 1 and 2 above 
on a medium customarily used for software interchange; or,
b) Accompany it with a written offer, valid for at least three years, to 
give any third party, for a charge no more than your cost of physically 
performing source distribution, a complete machine-readable copy of the 
corresponding source code, to be distributed under the terms of Sections 
1 and 2 above on a medium customarily used for software interchange; or,
c) Accompany it with the information you received as to the offer to 
distribute corresponding source code. (This alternative is allowed only 
for noncommercial distribution and only if you received the program in 
object code or executable form with such an offer, in accord with 
Subsection b above.)
The source code for a work means the preferred form of the work for 
making modifications to it. For an executable work, complete source code 
means all the source code for all modules it contains, plus any 
associated interface definition files, plus the scripts used to control 
compilation and installation of the executable. However, as a special 
exception, the source code distributed need not include anything that is 
normally distributed (in either source or binary form) with the major 
components (compiler, kernel, and so on) of the operating system on which 
the executable runs, unless that component itself accompanies the 
executable.

If distribution of executable or object code is made by offering access 
to copy from a designated place, then offering equivalent access to copy 
the source code from the same place counts as distribution of the source 
code, even though third parties are not compelled to copy the source 
along with the object code.

4. You may not copy, modify, sublicense, or distribute the Program except 
as expressly provided under this License. Any attempt otherwise to copy, 
modify, sublicense or distribute the Program is void, and will 
automatically terminate your rights under this License. However, parties 
who have received copies, or rights, from you under this License will not 
have their licenses terminated so long as such parties remain in full 
compliance.

5. You are not required to accept this License, since you have not signed 
it. However, nothing else grants you permission to modify or distribute 
the Program or its derivative works. These actions are prohibited by law 
if you do not accept this License. Therefore, by modifying or 
distributing the Program (or any work based on the Program), you indicate 
your acceptance of this License to do so, and all its terms and 
conditions for copying, distributing or modifying the Program or works 
based on it.

6. Each time you redistribute the Program (or any work based on the 
Program), the recipient automatically receives a license from the 
original licensor to copy, distribute or modify the Program subject to 
these terms and conditions. You may not impose any further restrictions 
on the recipients' exercise of the rights granted herein. You are not 
responsible for enforcing compliance by third parties to this License.

7. If, as a consequence of a court judgment or allegation of patent 
infringement or for any other reason (not limited to patent issues), 
conditions are imposed on you (whether by court order, agreement or 
otherwise) that contradict the conditions of this License, they do not 
excuse you from the conditions of this License. If you cannot distribute 
so as to satisfy simultaneously your obligations under this License and 
any other pertinent obligations, then as a consequence you may not 
distribute the Program at all. For example, if a patent license would not 
permit royalty-free redistribution of the Program by all those who 
receive copies directly or indirectly through you, then the only way you 
could satisfy both it and this License would be to refrain entirely from 
distribution of the Program.

If any portion of this section is held invalid or unenforceable under any 
particular circumstance, the balance of the section is intended to apply 
and the section as a whole is intended to apply in other circumstances.

It is not the purpose of this section to induce you to infringe any 
patents or other property right claims or to contest validity of any such 
claims; this section has the sole purpose of protecting the integrity of 
the free software distribution system, which is implemented by public 
license practices. Many people have made generous contributions to the 
wide range of software distributed through that system in reliance on 
consistent application of that system; it is up to the author/donor to 
decide if he or she is willing to distribute software through any other 
system and a licensee cannot impose that choice.

This section is intended to make thoroughly clear what is believed to be 
a consequence of the rest of this License.

8. If the distribution and/or use of the Program is restricted in certain 
countries either by patents or by copyrighted interfaces, the original 
copyright holder who places the Program under this License may add an 
explicit geographical distribution limitation excluding those countries, 
so that distribution is permitted only in or among countries not thus 
excluded. In such case, this License incorporates the limitation as if 
written in the body of this License.

9. The Free Software Foundation may publish revised and/or new versions 
of the General Public License from time to time. Such new versions will 
be similar in spirit to the present version, but may differ in detail to 
address new problems or concerns.

Each version is given a distinguishing version number. If the Program 
specifies a version number of this License which applies to it and "any 
later version", you have the option of following the terms and conditions 
either of that version or of any later version published by the Free 
Software Foundation. If the Program does not specify a version number of 
this License, you may choose any version ever published by the Free 
Software Foundation.

10. If you wish to incorporate parts of the Program into other free 
programs whose distribution conditions are different, write to the author 
to ask for permission. For software which is copyrighted by the Free 
Software Foundation, write to the Free Software Foundation; we sometimes 
make exceptions for this. Our decision will be guided by the two goals of 
preserving the free status of all derivatives of our free software and of 
promoting the sharing and reuse of software generally.

NO WARRANTY

11. BECAUSE THE PROGRAM IS LICENSED FREE OF CHARGE, THERE IS NO WARRANTY 
FOR THE PROGRAM, TO THE EXTENT PERMITTED BY APPLICABLE LAW. EXCEPT WHEN 
OTHERWISE STATED IN WRITING THE COPYRIGHT HOLDERS AND/OR OTHER PARTIES 
PROVIDE THE PROGRAM "AS IS" WITHOUT WARRANTY OF ANY KIND, EITHER 
EXPRESSED OR IMPLIED, INCLUDING, BUT NOT LIMITED TO, THE IMPLIED 
WARRANTIES OF MERCHANTABILITY AND FITNESS FOR A PARTICULAR PURPOSE. THE 
ENTIRE RISK AS TO THE QUALITY AND PERFORMANCE OF THE PROGRAM IS WITH YOU. 
SHOULD THE PROGRAM PROVE DEFECTIVE, YOU ASSUME THE COST OF ALL NECESSARY 
SERVICING, REPAIR OR CORRECTION.

12. IN NO EVENT UNLESS REQUIRED BY APPLICABLE LAW OR AGREED TO IN WRITING 
WILL ANY COPYRIGHT HOLDER, OR ANY OTHER PARTY WHO MAY MODIFY AND/OR 
REDISTRIBUTE THE PROGRAM AS PERMITTED ABOVE, BE LIABLE TO YOU FOR DAMAGES, 
INCLUDING ANY GENERAL, SPECIAL, INCIDENTAL OR CONSEQUENTIAL DAMAGES 
ARISING OUT OF THE USE OR INABILITY TO USE THE PROGRAM (INCLUDING BUT NOT 
LIMITED TO LOSS OF DATA OR DATA BEING RENDERED INACCURATE OR LOSSES 
SUSTAINED BY YOU OR THIRD PARTIES OR A FAILURE OF THE PROGRAM TO OPERATE 
WITH ANY OTHER PROGRAMS), EVEN IF SUCH HOLDER OR OTHER PARTY HAS BEEN 
ADVISED OF THE POSSIBILITY OF SUCH DAMAGES.

END OF TERMS AND CONDITIONS

How to Apply These Terms to Your New Programs

If you develop a new program, and you want it to be of the greatest 
possible use to the public, the best way to achieve this is to make it 
free software which everyone can redistribute and change under these 
terms.

To do so, attach the following notices to the program. It is safest to 
attach them to the start of each source file to most effectively convey 
the exclusion of warranty; and each file should have at least the 
"copyright" line and a pointer to where the full notice is found.

one line to give the program's name and an idea of what it does.
Copyright (C) yyyy  name of author

This program is free software; you can redistribute it and/or
modify it under the terms of the GNU General Public License
as published by the Free Software Foundation; either version 2
of the License, or (at your option) any later version.

This program is distributed in the hope that it will be useful,
but WITHOUT ANY WARRANTY; without even the implied warranty of
MERCHANTABILITY or FITNESS FOR A PARTICULAR PURPOSE.  See the
GNU General Public License for more details.

You should have received a copy of the GNU General Public License
along with this program; if not, write to the Free Software
Foundation, Inc., 51 Franklin Street, Fifth Floor, Boston, MA  02110-1301, 
USA.
Also add information on how to contact you by electronic and paper mail.

If the program is interactive, make it output a short notice like this 
when it starts in an interactive mode:

Gnomovision version 69, Copyright (C) year name of author
Gnomovision comes with ABSOLUTELY NO WARRANTY; for details
type `show w'.  This is free software, and you are welcome
to redistribute it under certain conditions; type `show c' 
for details.
The hypothetical commands `show w' and `show c' should show the 
appropriate parts of the General Public License. Of course, the commands 
you use may be called something other than `show w' and `show c'; they 
could even be mouse-clicks or menu items--whatever suits your program.

You should also get your employer (if you work as a programmer) or your 
school, if any, to sign a "copyright disclaimer" for the program, if 
necessary. Here is a sample; alter the names:

Yoyodyne, Inc., hereby disclaims all copyright
interest in the program `Gnomovision'
(which makes passes at compilers) written 
by James Hacker.

signature of Ty Coon, 1 April 1989
Ty Coon, President of Vice
This General Public License does not permit incorporating your program 
into proprietary programs. If your program is a subroutine library, you 
may consider it more useful to permit linking proprietary applications 
with the library. If this is what you want to do, use the GNU Lesser 
General Public License instead of this License.
\end{Verbatim}

%----------------------------------------------------------------------------
\subsection{GNU GPLv3}
%----------------------------------------------------------------------------

\begin{Verbatim}[fontsize=\footnotesize]
TERMS AND CONDITIONS

0. Definitions.

"This License" refers to version 3 of the GNU General Public License.

"Copyright" also means copyright-like laws that apply to other kinds of 
works, such as semiconductor masks.

"The Program" refers to any copyrightable work licensed under this 
License. Each licensee is addressed as "you". "Licensees" and "recipients" 
may be individuals or organizations.

To "modify" a work means to copy from or adapt all or part of the work in 
a fashion requiring copyright permission, other than the making of an 
exact copy. The resulting work is called a "modified version" of the 
earlier work or a work "based on" the earlier work.

A "covered work" means either the unmodified Program or a work based on 
the Program.

To "propagate" a work means to do anything with it that, without 
permission, would make you directly or secondarily liable for 
infringement under applicable copyright law, except executing it on a 
computer or modifying a private copy. Propagation includes copying, 
distribution (with or without modification), making available to the 
public, and in some countries other activities as well.

To "convey" a work means any kind of propagation that enables other 
parties to make or receive copies. Mere interaction with a user through a 
computer network, with no transfer of a copy, is not conveying.

An interactive user interface displays "Appropriate Legal Notices" to the 
extent that it includes a convenient and prominently visible feature that 
(1) displays an appropriate copyright notice, and (2) tells the user that 
there is no warranty for the work (except to the extent that warranties 
are provided), that licensees may convey the work under this License, and 
how to view a copy of this License. If the interface presents a list of 
user commands or options, such as a menu, a prominent item in the list 
meets this criterion.

1. Source Code.

The "source code" for a work means the preferred form of the work for 
making modifications to it. "Object code" means any non-source form of a 
work.

A "Standard Interface" means an interface that either is an official 
standard defined by a recognized standards body, or, in the case of 
interfaces specified for a particular programming language, one that is 
widely used among developers working in that language.

The "System Libraries" of an executable work include anything, other than 
the work as a whole, that (a) is included in the normal form of packaging 
a Major Component, but which is not part of that Major Component, and (b) 
serves only to enable use of the work with that Major Component, or to 
implement a Standard Interface for which an implementation is available 
to the public in source code form. A "Major Component", in this context, 
means a major essential component (kernel, window system, and so on) of 
the specific operating system (if any) on which the executable work runs, 
or a compiler used to produce the work, or an object code interpreter 
used to run it.

The "Corresponding Source" for a work in object code form means all the 
source code needed to generate, install, and (for an executable work) run 
the object code and to modify the work, including scripts to control 
those activities. However, it does not include the work's System 
Libraries, or general-purpose tools or generally available free programs 
which are used unmodified in performing those activities but which are 
not part of the work. For example, Corresponding Source includes 
interface definition files associated with source files for the work, and 
the source code for shared libraries and dynamically linked subprograms 
that the work is specifically designed to require, such as by intimate 
data communication or control flow between those subprograms and other 
parts of the work.

The Corresponding Source need not include anything that users can 
regenerate automatically from other parts of the Corresponding Source.

The Corresponding Source for a work in source code form is that same work.

2. Basic Permissions.

All rights granted under this License are granted for the term of 
copyright on the Program, and are irrevocable provided the stated 
conditions are met. This License explicitly affirms your unlimited 
permission to run the unmodified Program. The output from running a 
covered work is covered by this License only if the output, given its 
content, constitutes a covered work. This License acknowledges your 
rights of fair use or other equivalent, as provided by copyright law.

You may make, run and propagate covered works that you do not convey, 
without conditions so long as your license otherwise remains in force. 
You may convey covered works to others for the sole purpose of having 
them make modifications exclusively for you, or provide you with 
facilities for running those works, provided that you comply with the 
terms of this License in conveying all material for which you do not 
control copyright. Those thus making or running the covered works for you 
must do so exclusively on your behalf, under your direction and control, 
on terms that prohibit them from making any copies of your copyrighted 
material outside their relationship with you.

Conveying under any other circumstances is permitted solely under the 
conditions stated below. Sublicensing is not allowed; section 10 makes it 
unnecessary.

3. Protecting Users' Legal Rights From Anti-Circumvention Law.

No covered work shall be deemed part of an effective technological 
measure under any applicable law fulfilling obligations under article 11 
of the WIPO copyright treaty adopted on 20 December 1996, or similar laws 
prohibiting or restricting circumvention of such measures.

When you convey a covered work, you waive any legal power to forbid 
circumvention of technological measures to the extent such circumvention 
is effected by exercising rights under this License with respect to the 
covered work, and you disclaim any intention to limit operation or 
modification of the work as a means of enforcing, against the work's 
users, your or third parties' legal rights to forbid circumvention of 
technological measures.

4. Conveying Verbatim Copies.

You may convey verbatim copies of the Program's source code as you 
receive it, in any medium, provided that you conspicuously and 
appropriately publish on each copy an appropriate copyright notice; keep 
intact all notices stating that this License and any non-permissive terms 
added in accord with section 7 apply to the code; keep intact all notices 
of the absence of any warranty; and give all recipients a copy of this 
License along with the Program.

You may charge any price or no price for each copy that you convey, and 
you may offer support or warranty protection for a fee.

5. Conveying Modified Source Versions.

You may convey a work based on the Program, or the modifications to 
produce it from the Program, in the form of source code under the terms 
of section 4, provided that you also meet all of these conditions:

a) The work must carry prominent notices stating that you modified it, 
and giving a relevant date.
b) The work must carry prominent notices stating that it is released 
under this License and any conditions added under section 7. This 
requirement modifies the requirement in section 4 to "keep intact all 
notices".
c) You must license the entire work, as a whole, under this License to 
anyone who comes into possession of a copy. This License will therefore 
apply, along with any applicable section 7 additional terms, to the whole 
of the work, and all its parts, regardless of how they are packaged. This 
License gives no permission to license the work in any other way, but it 
does not invalidate such permission if you have separately received it.
d) If the work has interactive user interfaces, each must display 
Appropriate Legal Notices; however, if the Program has interactive 
interfaces that do not display Appropriate Legal Notices, your work need 
not make them do so.
A compilation of a covered work with other separate and independent works, 
which are not by their nature extensions of the covered work, and which 
are not combined with it such as to form a larger program, in or on a 
volume of a storage or distribution medium, is called an "aggregate" if 
the compilation and its resulting copyright are not used to limit the 
access or legal rights of the compilation's users beyond what the 
individual works permit. Inclusion of a covered work in an aggregate does 
not cause this License to apply to the other parts of the aggregate.

6. Conveying Non-Source Forms.

You may convey a covered work in object code form under the terms of 
sections 4 and 5, provided that you also convey the machine-readable 
Corresponding Source under the terms of this License, in one of these 
ways:

a) Convey the object code in, or embodied in, a physical product 
(including a physical distribution medium), accompanied by the 
Corresponding Source fixed on a durable physical medium customarily used 
for software interchange.
b) Convey the object code in, or embodied in, a physical product 
(including a physical distribution medium), accompanied by a written 
offer, valid for at least three years and valid for as long as you offer 
spare parts or customer support for that product model, to give anyone 
who possesses the object code either (1) a copy of the Corresponding 
Source for all the software in the product that is covered by this 
License, on a durable physical medium customarily used for software 
interchange, for a price no more than your reasonable cost of physically 
performing this conveying of source, or (2) access to copy the 
Corresponding Source from a network server at no charge.
c) Convey individual copies of the object code with a copy of the 
written offer to provide the Corresponding Source. This alternative is 
allowed only occasionally and noncommercially, and only if you received 
the object code with such an offer, in accord with subsection 6b.
d) Convey the object code by offering access from a designated place 
(gratis or for a charge), and offer equivalent access to the 
Corresponding Source in the same way through the same place at no 
further charge. You need not require recipients to copy the 
Corresponding Source along with the object code. If the place to copy 
the object code is a network server, the Corresponding Source may be on 
a different server (operated by you or a third party) that supports 
equivalent copying facilities, provided you maintain clear directions 
next to the object code saying where to find the Corresponding Source. 
Regardless of what server hosts the Corresponding Source, you remain 
obligated to ensure that it is available for as long as needed to 
satisfy these requirements.
e) Convey the object code using peer-to-peer transmission, provided you 
inform other peers where the object code and Corresponding Source of the 
work are being offered to the general public at no charge under 
subsection 6d.
A separable portion of the object code, whose source code is excluded 
from the Corresponding Source as a System Library, need not be included 
in conveying the object code work.

A "User Product" is either (1) a "consumer product", which means any 
tangible personal property which is normally used for personal, family, 
or household purposes, or (2) anything designed or sold for 
incorporation into a dwelling. In determining whether a product is a 
consumer product, doubtful cases shall be resolved in favor of coverage. 
For a particular product received by a particular user, "normally used" 
refers to a typical or common use of that class of product, regardless 
of the status of the particular user or of the way in which the 
particular user actually uses, or expects or is expected to use, the 
product. A product is a consumer product regardless of whether the 
product has substantial commercial, industrial or non-consumer uses, 
unless such uses represent the only significant mode of use of the 
product.

"Installation Information" for a User Product means any methods, 
procedures, authorization keys, or other information required to install 
and execute modified versions of a covered work in that User Product 
from a modified version of its Corresponding Source. The information 
must suffice to ensure that the continued functioning of the modified 
object code is in no case prevented or interfered with solely because 
modification has been made.

If you convey an object code work under this section in, or with, or 
specifically for use in, a User Product, and the conveying occurs as 
part of a transaction in which the right of possession and use of the 
User Product is transferred to the recipient in perpetuity or for a 
fixed term (regardless of how the transaction is characterized), the 
Corresponding Source conveyed under this section must be accompanied by 
the Installation Information. But this requirement does not apply if 
neither you nor any third party retains the ability to install modified 
object code on the User Product (for example, the work has been 
installed in ROM).

The requirement to provide Installation Information does not include a 
requirement to continue to provide support service, warranty, or updates 
for a work that has been modified or installed by the recipient, or for 
the User Product in which it has been modified or installed. Access to a 
network may be denied when the modification itself materially and 
adversely affects the operation of the network or violates the rules and 
protocols for communication across the network.

Corresponding Source conveyed, and Installation Information provided, in 
accord with this section must be in a format that is publicly documented 
(and with an implementation available to the public in source code form), 
and must require no special password or key for unpacking, reading or 
copying.

7. Additional Terms.

"Additional permissions" are terms that supplement the terms of this 
License by making exceptions from one or more of its conditions. 
Additional permissions that are applicable to the entire Program shall 
be treated as though they were included in this License, to the extent 
that they are valid under applicable law. If additional permissions 
apply only to part of the Program, that part may be used separately 
under those permissions, but the entire Program remains governed by this 
License without regard to the additional permissions.

When you convey a copy of a covered work, you may at your option remove 
any additional permissions from that copy, or from any part of it. 
(Additional permissions may be written to require their own removal in 
certain cases when you modify the work.) You may place additional 
permissions on material, added by you to a covered work, for which you 
have or can give appropriate copyright permission.

Notwithstanding any other provision of this License, for material you 
add to a covered work, you may (if authorized by the copyright holders 
of that material) supplement the terms of this License with terms:

a) Disclaiming warranty or limiting liability differently from the terms 
of sections 15 and 16 of this License; or
b) Requiring preservation of specified reasonable legal notices or 
author attributions in that material or in the Appropriate Legal Notices 
displayed by works containing it; or
c) Prohibiting misrepresentation of the origin of that material, or 
requiring that modified versions of such material be marked in 
reasonable ways as different from the original version; or
d) Limiting the use for publicity purposes of names of licensors or 
authors of the material; or
e) Declining to grant rights under trademark law for use of some trade 
names, trademarks, or service marks; or
f) Requiring indemnification of licensors and authors of that material 
by anyone who conveys the material (or modified versions of it) with 
contractual assumptions of liability to the recipient, for any liability 
that these contractual assumptions directly impose on those licensors 
and authors.
All other non-permissive additional terms are considered "further 
restrictions" within the meaning of section 10. If the Program as you 
received it, or any part of it, contains a notice stating that it is 
governed by this License along with a term that is a further restriction, 
you may remove that term. If a license document contains a further 
restriction but permits relicensing or conveying under this License, you 
may add to a covered work material governed by the terms of that license 
document, provided that the further restriction does not survive such 
relicensing or conveying.

If you add terms to a covered work in accord with this section, you must 
place, in the relevant source files, a statement of the additional terms 
that apply to those files, or a notice indicating where to find the 
applicable terms.

Additional terms, permissive or non-permissive, may be stated in the 
form of a separately written license, or stated as exceptions; the above 
requirements apply either way.

8. Termination.

You may not propagate or modify a covered work except as expressly 
provided under this License. Any attempt otherwise to propagate or 
modify it is void, and will automatically terminate your rights under 
this License (including any patent licenses granted under the third 
paragraph of section 11).

However, if you cease all violation of this License, then your license 
from a particular copyright holder is reinstated (a) provisionally, 
unless and until the copyright holder explicitly and finally terminates 
your license, and (b) permanently, if the copyright holder fails to 
notify you of the violation by some reasonable means prior to 60 days 
after the cessation.

Moreover, your license from a particular copyright holder is reinstated 
permanently if the copyright holder notifies you of the violation by 
some reasonable means, this is the first time you have received notice 
of violation of this License (for any work) from that copyright holder, 
and you cure the violation prior to 30 days after your receipt of the 
notice.

Termination of your rights under this section does not terminate the 
licenses of parties who have received copies or rights from you under 
this License. If your rights have been terminated and not permanently 
reinstated, you do not qualify to receive new licenses for the same 
material under section 10.

9. Acceptance Not Required for Having Copies.

You are not required to accept this License in order to receive or run a 
copy of the Program. Ancillary propagation of a covered work occurring 
solely as a consequence of using peer-to-peer transmission to receive a 
copy likewise does not require acceptance. However, nothing other than 
this License grants you permission to propagate or modify any covered 
work. These actions infringe copyright if you do not accept this 
License. Therefore, by modifying or propagating a covered work, you 
indicate your acceptance of this License to do so.

10. Automatic Licensing of Downstream Recipients.

Each time you convey a covered work, the recipient automatically 
receives a license from the original licensors, to run, modify and 
propagate that work, subject to this License. You are not responsible 
for enforcing compliance by third parties with this License.

An "entity transaction" is a transaction transferring control of an 
organization, or substantially all assets of one, or subdividing an 
organization, or merging organizations. If propagation of a covered work 
results from an entity transaction, each party to that transaction who 
receives a copy of the work also receives whatever licenses to the work 
the party's predecessor in interest had or could give under the previous 
paragraph, plus a right to possession of the Corresponding Source of the 
work from the predecessor in interest, if the predecessor has it or can 
get it with reasonable efforts.

You may not impose any further restrictions on the exercise of the 
rights granted or affirmed under this License. For example, you may not 
impose a license fee, royalty, or other charge for exercise of rights 
granted under this License, and you may not initiate litigation 
(including a cross-claim or counterclaim in a lawsuit) alleging that any 
patent claim is infringed by making, using, selling, offering for sale, 
or importing the Program or any portion of it.

11. Patents.

A "contributor" is a copyright holder who authorizes use under this 
License of the Program or a work on which the Program is based. The work 
thus licensed is called the contributor's "contributor version".

A contributor's "essential patent claims" are all patent claims owned or 
controlled by the contributor, whether already acquired or hereafter 
acquired, that would be infringed by some manner, permitted by this 
License, of making, using, or selling its contributor version, but do 
not include claims that would be infringed only as a consequence of 
further modification of the contributor version. For purposes of this 
definition, "control" includes the right to grant patent sublicenses in 
a manner consistent with the requirements of this License.

Each contributor grants you a non-exclusive, worldwide, royalty-free 
patent license under the contributor's essential patent claims, to make, 
use, sell, offer for sale, import and otherwise run, modify and 
propagate the contents of its contributor version.

In the following three paragraphs, a "patent license" is any express 
agreement or commitment, however denominated, not to enforce a patent 
(such as an express permission to practice a patent or covenant not to 
sue for patent infringement). To "grant" such a patent license to a 
party means to make such an agreement or commitment not to enforce a 
patent against the party.

If you convey a covered work, knowingly relying on a patent license, and 
the Corresponding Source of the work is not available for anyone to copy, 
free of charge and under the terms of this License, through a publicly 
available network server or other readily accessible means, then you 
must either (1) cause the Corresponding Source to be so available, or (2) 
arrange to deprive yourself of the benefit of the patent license for 
this particular work, or (3) arrange, in a manner consistent with the 
requirements of this License, to extend the patent license to downstream 
recipients. "Knowingly relying" means you have actual knowledge that, 
but for the patent license, your conveying the covered work in a country, 
or your recipient's use of the covered work in a country, would infringe 
one or more identifiable patents in that country that you have reason to 
believe are valid.

If, pursuant to or in connection with a single transaction or 
arrangement, you convey, or propagate by procuring conveyance of, a 
covered work, and grant a patent license to some of the parties 
receiving the covered work authorizing them to use, propagate, modify or 
convey a specific copy of the covered work, then the patent license you 
grant is automatically extended to all recipients of the covered work 
and works based on it.

A patent license is "discriminatory" if it does not include within the 
scope of its coverage, prohibits the exercise of, or is conditioned on 
the non-exercise of one or more of the rights that are specifically 
granted under this License. You may not convey a covered work if you are 
a party to an arrangement with a third party that is in the business of 
distributing software, under which you make payment to the third party 
based on the extent of your activity of conveying the work, and under 
which the third party grants, to any of the parties who would receive 
the covered work from you, a discriminatory patent license (a) in 
connection with copies of the covered work conveyed by you (or copies 
made from those copies), or (b) primarily for and in connection with 
specific products or compilations that contain the covered work, unless 
you entered into that arrangement, or that patent license was granted, 
prior to 28 March 2007.

Nothing in this License shall be construed as excluding or limiting any 
implied license or other defenses to infringement that may otherwise be 
available to you under applicable patent law.

12. No Surrender of Others' Freedom.

If conditions are imposed on you (whether by court order, agreement or 
otherwise) that contradict the conditions of this License, they do not 
excuse you from the conditions of this License. If you cannot convey a 
covered work so as to satisfy simultaneously your obligations under this 
License and any other pertinent obligations, then as a consequence you 
may not convey it at all. For example, if you agree to terms that 
obligate you to collect a royalty for further conveying from those to 
whom you convey the Program, the only way you could satisfy both those 
terms and this License would be to refrain entirely from conveying the 
Program.

13. Use with the GNU Affero General Public License.

Notwithstanding any other provision of this License, you have permission 
to link or combine any covered work with a work licensed under version 3 
of the GNU Affero General Public License into a single combined work, 
and to convey the resulting work. The terms of this License will 
continue to apply to the part which is the covered work, but the special 
requirements of the GNU Affero General Public License, section 13, 
concerning interaction through a network will apply to the combination 
as such.

14. Revised Versions of this License.

The Free Software Foundation may publish revised and/or new versions of 
the GNU General Public License from time to time. Such new versions will 
be similar in spirit to the present version, but may differ in detail to 
address new problems or concerns.

Each version is given a distinguishing version number. If the Program 
specifies that a certain numbered version of the GNU General Public 
License "or any later version" applies to it, you have the option of 
following the terms and conditions either of that numbered version or of 
any later version published by the Free Software Foundation. If the 
Program does not specify a version number of the GNU General Public 
License, you may choose any version ever published by the Free Software 
Foundation.

If the Program specifies that a proxy can decide which future versions 
of the GNU General Public License can be used, that proxy's public 
statement of acceptance of a version permanently authorizes you to 
choose that version for the Program.

Later license versions may give you additional or different permissions. 
However, no additional obligations are imposed on any author or 
copyright holder as a result of your choosing to follow a later version.

15. Disclaimer of Warranty.

THERE IS NO WARRANTY FOR THE PROGRAM, TO THE EXTENT PERMITTED BY 
APPLICABLE LAW. EXCEPT WHEN OTHERWISE STATED IN WRITING THE COPYRIGHT 
HOLDERS AND/OR OTHER PARTIES PROVIDE THE PROGRAM "AS IS" WITHOUT 
WARRANTY OF ANY KIND, EITHER EXPRESSED OR IMPLIED, INCLUDING, BUT NOT 
LIMITED TO, THE IMPLIED WARRANTIES OF MERCHANTABILITY AND FITNESS FOR A 
PARTICULAR PURPOSE. THE ENTIRE RISK AS TO THE QUALITY AND PERFORMANCE OF 
THE PROGRAM IS WITH YOU. SHOULD THE PROGRAM PROVE DEFECTIVE, YOU ASSUME 
THE COST OF ALL NECESSARY SERVICING, REPAIR OR CORRECTION.

16. Limitation of Liability.

IN NO EVENT UNLESS REQUIRED BY APPLICABLE LAW OR AGREED TO IN WRITING 
WILL ANY COPYRIGHT HOLDER, OR ANY OTHER PARTY WHO MODIFIES AND/OR 
CONVEYS THE PROGRAM AS PERMITTED ABOVE, BE LIABLE TO YOU FOR DAMAGES, 
INCLUDING ANY GENERAL, SPECIAL, INCIDENTAL OR CONSEQUENTIAL DAMAGES 
ARISING OUT OF THE USE OR INABILITY TO USE THE PROGRAM (INCLUDING BUT 
NOT LIMITED TO LOSS OF DATA OR DATA BEING RENDERED INACCURATE OR LOSSES 
SUSTAINED BY YOU OR THIRD PARTIES OR A FAILURE OF THE PROGRAM TO OPERATE 
WITH ANY OTHER PROGRAMS), EVEN IF SUCH HOLDER OR OTHER PARTY HAS BEEN 
ADVISED OF THE POSSIBILITY OF SUCH DAMAGES.

17. Interpretation of Sections 15 and 16.

If the disclaimer of warranty and limitation of liability provided above 
cannot be given local legal effect according to their terms, reviewing 
courts shall apply local law that most closely approximates an absolute 
waiver of all civil liability in connection with the Program, unless a 
warranty or assumption of liability accompanies a copy of the Program in 
return for a fee.

END OF TERMS AND CONDITIONS

How to Apply These Terms to Your New Programs

If you develop a new program, and you want it to be of the greatest 
possible use to the public, the best way to achieve this is to make it 
free software which everyone can redistribute and change under these 
terms.

To do so, attach the following notices to the program. It is safest to 
attach them to the start of each source file to most effectively state 
the exclusion of warranty; and each file should have at least the 
"copyright" line and a pointer to where the full notice is found.

    <one line to give the program's name and a brief idea of what it 
    does.>
    Copyright (C) <year>  <name of author>

    This program is free software: you can redistribute it and/or modify
    it under the terms of the GNU General Public License as published by
    the Free Software Foundation, either version 3 of the License, or
    (at your option) any later version.

    This program is distributed in the hope that it will be useful,
    but WITHOUT ANY WARRANTY; without even the implied warranty of
    MERCHANTABILITY or FITNESS FOR A PARTICULAR PURPOSE.  See the
    GNU General Public License for more details.

    You should have received a copy of the GNU General Public License
    along with this program.  If not, see <http://www.gnu.org/licenses/>.
Also add information on how to contact you by electronic and paper mail.

If the program does terminal interaction, make it output a short notice 
like this when it starts in an interactive mode:

    <program>  Copyright (C) <year>  <name of author>
    This program comes with ABSOLUTELY NO WARRANTY; for details type 
    `show w'.
    This is free software, and you are welcome to redistribute it
    under certain conditions; type `show c' for details.
The hypothetical commands `show w' and `show c' should show the 
appropriate parts of the General Public License. Of course, your 
program's commands might be different; for a GUI interface, you would 
use an "about box".

You should also get your employer (if you work as a programmer) or 
school, if any, to sign a "copyright disclaimer" for the program, if 
necessary. For more information on this, and how to apply and follow the 
GNU GPL, see <http://www.gnu.org/licenses/>.

The GNU General Public License does not permit incorporating your 
program into proprietary programs. If your program is a subroutine 
library, you may consider it more useful to permit linking proprietary 
applications with the library. If this is what you want to do, use the 
GNU Lesser General Public License instead of this License. But first, 
please read <http://www.gnu.org/philosophy/why-not-lgpl.html>.
\end{Verbatim}